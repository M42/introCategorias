\documentclass[a4paper, 11pt]{amsart}
\usepackage{amssymb}

%%% Castellano
\usepackage[spanish]{babel} 
\selectlanguage{spanish}
\usepackage[utf8]{inputenc}

%%% Diagramas
% Conmutative diagrams
\usepackage{tkz-graph}
\usetikzlibrary{arrows}

%%% Shortcuts
\newcommand{\C}{\mathcal{C} }

%%% Twopart definition
\newcommand{\twopartdef}[4]
{
	\left\{
		\begin{array}{ll}
			#1 & \mbox{si } #2 \\
			#3 & \mbox{si } #4
		\end{array}
	\right.
}



\newtheorem{theorem}{Theorem}[section]
\newtheorem{lemma}[theorem]{Lemma}

\theoremstyle{definition}
\newtheorem{definition}[theorem]{Definición}
\newtheorem{example}[theorem]{Ejemplo}
\newtheorem{exca}[theorem]{Ejercicio}

\theoremstyle{remark}
\newtheorem{remark}[theorem]{Remark}

\numberwithin{equation}{section}

\begin{document}

\title{Introducción a la teoría de categorías}

%    Remove any unused author tags.

%    author one information
\author{Mario Román}
\address{}
\curraddr{}
\email{}
\thanks{}

% References
%\subjclass[2000]{Primary }
%    For articles to be published after 1 January 2010, you may use
%    the following version:
%\subjclass[2010]{Primary }

\keywords{}

\date{}

\dedicatory{}

\begin{abstract}
  Presentamos una breve introducción a la teoría de categorías, motivando y definiendo el concepto.
  Se indican los primeros ejemplos de categorías.
\end{abstract}

\maketitle

\section {Categorías}
  \subsection {Motivación}
    Varias estructuras matemáticas (grupos, espacios vectoriales, espacios topológicos \dots) cuentan
    con morfismos que preservan las estructura subyacentes entre ellas. Como ejemplos:
    \begin {center}
    \begin{tabular}{l|l}
      Conjunto & Morfismos \\
      \hline
      Grupos & Homomorfismos de grupos \\
      Espacios topológicos & Funciones continuas \\
      Espacios métricos & Funciones cortas \\
      Conjuntos & Funciones \\
      Espacios vectoriales sobre $\mathbb{K}$ & Funciones lineales sobre $\mathbb{K}$ \\
    \end{tabular}
    \end{center}
    Si estudiamos axiomáticamente las propiedades abstractas de estas estructuras y sus morfismos,
    obtendremos teoremas particularizables a todos estos casos útiles por sí mismos.
    Una categoría la formarán una clase de estos espacios con estructura y los morfismos entre estos
    espacios; y los teoremas que deduzcamos para todas las categorías podrán aplicarse a cada uno de
    los espacios.
    
  \subsection {Definición formal}
    \definition
    Una \textbf{categoría} $\C$ está definida por:
    \begin{itemize}
      \item Una clase de objetos de la categoría, $Obj(\mathcal{C})$.
      \item Un conjunto de morfismos $Hom_{\C}(A,B)$, poblado o no, entre cada par de objetos $A,B \in Obj(\C)$.
    \end{itemize}

    Cumpliendo sus morfismos las siguientes propiedades:
    \begin{itemize}
      \item Para dos morfismos $f \in Hom(A,B)$, $g \in Hom(B,C)$, existe su morfismo composición $f \circ g$.
      \item La composición es asociativa: $ f \circ (g \circ h) = (f \circ g) \circ h$
      \item Todos los objetos tienen un morfismo identidad, $1_{A} \in Hom(A,A)$, 
	  neutro para la composición: $\forall f \in Hom(A,B): f \circ 1_{A} = 1_{B} \circ f = f$
    \end{itemize}
    
    \exca Demostrar que la identidad es el único elemento neutro para la composición.

  \subsection {Ejemplos}
    Como idea simplificadora, podemos que los objetos son conjuntos, y los morfismos, funciones
    entre esos conjuntos; de hecho, el primer ejemplo es la definición de ese caso concreto. 
    Este es un buen modelo intuitivo para trabajar con algunas categorías,
    pero presentaremos ejemplos que rechazan esta intuición.
    \subsubsection{Categoría \texttt{Set}}
      La categoría de los conjuntos con las funciones entre conjuntos como morfismos.
      \begin{gather*}
        Obj(\texttt{Set}) = \{Todos\ los\ conjuntos\} \\
        Hom(A,B)= B^A = \{f \;|\; f: A \rightarrow B \}
      \end{gather*}
      Trivialmente es categoría: las funciones se componen en funciones, la composición es
      asociativa y la identidad funciona como se espera, siendo otra función.
      
    \subsubsection{Categoría \texttt{VectR}}
      La categoría de los espacios vectoriales reales con las funciones lineales entre
      espacios vectoriales reales.
      \begin{gather*}
        Obj(\texttt{VectR}) = \{Todos\ los\ espacios\ vectoriales\ sobre\ \mathbb{R}\} \\
        Hom(A,B)= \mathcal{L}_{\mathbb{R}}(A,B)
      \end{gather*}
      \exca Observar que \texttt{VectR} es una categoría.
    
    \subsubsection{Categoría \texttt{$(S,\sim)$}}
      Cualquier conjunto $S$ que tenga definida una relación de equivalencia $\sim$ tiene
      definida una categoría asociada en la que los objetos son los elementos del conjunto
      y los morfismos sólo representan casos particulares de la relación de equivalencia.
      \textit{En este ejemplo, los morfismos no son funciones y los objetos no son conjuntos,
      rechazando por primera vez la intuición del primer ejemplo.}
      \begin{gather*}
        Obj((S,\sim)) = S
      \end{gather*}
      Hay un morfismo entre dos elementos si y sólo si están relacionados:
      \begin{align*}
        Hom(a,b)= \twopartdef{(a,b)}{a \sim b}{\emptyset}{a \nsim b}
      \end{align*}
      Y la composición se define como:
      \begin{align*}
       (a,b) \circ (b,c) = (a,c)
      \end{align*}
      Probar que es categoría se reduce a notar que la composición de morfismos es morfismo (por
      ser la relación transitiva), que la composición es asociativa y que existe el morfismo identidad
      $(a,a)$, por ser la relación reflexiva.
      
    
% \section {Tipos de morfismos}
%   \subsection {Isomorfismos}
%   \subsection {Epimorfismos y monomorfismos}
% 
% \section {Propiedades universales}
% \subsection {Objetos iniciales}
% \subsection {Objetos finales}
% \subsection {Ejemplos}
% \subsection {Propiedades universales de map, filter, fold}
% 
% \section {Productos y coproductos}
% 
% \section {Functores}
% 
% \section {Transformaciones naturales}

\end{document}

%-----------------------------------------------------------------------
% End of amsart.template
%-----------------------------------------------------------------------
